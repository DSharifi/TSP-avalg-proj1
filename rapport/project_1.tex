\documentclass[a4paper,12pt]{article}
\usepackage{srcltx}
\usepackage[english,swedish]{babel}
\usepackage[utf8]{inputenc}
\usepackage[T1]{fontenc}
\usepackage{parskip}
\usepackage{amsmath}
\usepackage{algorithmic}


\renewcommand{\O}{\ensuremath{\mathcal{O}}}
\renewcommand{\*}{\ensuremath{\cdot}}

\begin{document}

\selectlanguage{english}
\title{DD2440 Advanced Algorithms (fall 2009) \\ Project 1 -- Euclidian TSP}
\author{Joel Pettersson \\ 880519-0637 \\ joelpet@kth.se \and Linus Wallgren \\
880213-0099  \\ linuswa@kth.se}
\date{\today}
\maketitle
\thispagestyle{empty}
\newpage
\setcounter{page}{1}

\selectlanguage{swedish}

\section{Inledning}

% TODO TSP har massa tillämpningar och är därför intressant

\subsection{Problembeskrivning} 
%Beskriv TSP, gärna lite snyggt formellt med fina matematiska notationer. :)

Travelling Salesman Problem (TSP), också kallat handelsresandeproblemet, går ut
på att hitta den kortaste hamiltoncykeln\footnote{En hamiltoncykel är en stig i
en graf som passerar samtliga noder precis en gång, med undantag för noden där
stigen startar och slutar.} i en komplett graf. Det är väl känt att TSP tillhör
de NP-fullständiga problemen, och därmed ej kan förväntas lösas i polynomisk
tid. Av den anledningen kommer denna rapport fokusera på olika sätt att ta fram
approximativa lösningar.

Problemet framträder i en rad olika skepnader, där var och en innehåller
skiftande restriktioner. Den variant av TSP som här behandlas beskrivs genom
punkter i planet med euklidiska avstånd och därigenom är även triangelolikheten
uppfylld.

Låt $d_{ij}$ beteckna avståndet mellan städerna $t_i$ och $t_j$. Då innebär
triangelolikheten att $d_{ik} \leq d_{ij} + d_{jk}$ gäller. Det kan uttryckas
som att sträckan mellan två städer $t_i$ och $t_k$ aldrig kan bli kortare genom
att passera en annan stad $t_j$ på vägen. I specialfallet då sträckan blir lika
lång ligger städerna på en rät linje i planet. Målet är alltså att utifrån dessa
förutsättningar minimera utrycket i (\ref{eq:tourSum}), där $n$ är det totala
antalet noder i grafen.

\begin{align}
    \sum_{i=0}^{n}{\sum_{j=0}^{n}{x_{ij} \cdot d_{ij}}},  & 
    & x_{ij} =
    \begin{cases}
        1 & (i,j) \textrm{ tillhör turen} \\
        0 & \textrm{annars}
    \end{cases}
    \label{eq:tourSum}
\end{align}

%TODO lägg till beskrivning av x_i,j  med case
%där 
%d_{i,j} är avståndet mellan nod i och nod j.
%x_{i,j} är 1 om kanten (i,j) är del av turen och 0 annars.

Slutligen måste turen beskriven av $x_{ij}$ bilda en hamiltoncykel.

\subsection{Metod}

Här ges en kort sammanfattning av vilka metoder vi har valt att använda.
Algoritmerna som nämns i följande stycke kommer att presenteras mer detaljerat
senare i rapporten.

Inledningsvis implementerade vi den naiva startheuristiken som låter noderna
vara sorterade från $1$ till $n$. På den turen applicerade vi i nästa steg en
lokal sökning, närmare bestämt 2-opt. När den lokala sökningen hade konstaterats
fungera korrekt ersatte vi startheuristiken med en Nearest
neighbor-konstruktion. Slutligen fick 2-opt kompletteras med 2.5-opt. Från
början var det dessutom tänkt att vi skulle hinna implementera 3-opt, vilket
dock ej blev fallet.

Ovanstående val av tillvägagångssätt och algoritmer baserades på de empiriska
resultat som presenteras i \textit{Notes for the course advanced algorithms}
\footnote{http://www.csc.kth.se/utbildning/kth/kurser/DD2440/avalg07/algnotes.pdf}
på sidorna 133 och 135. Dessa tydde förvisso på ett övertag för 3-opt och
framför allt för Lin-Kernighan med avseende på turernas kvalitet, men krävde i
gengäld längre körningstid. Vidare bedömde vi Lin-Kernighan som jämförelsevis
intrikat att implementera korrekt.

\section{Konstruktionsheuristiker}

\subsection{Nearest neighbor}

Nearest Neighbor fungerar precis som namnet antyder; från en slumpmässig
startnod läggs hela tiden den närmsta noden till.

\begin{algorithmic}
    \STATE $N$ := Nodes()
    \STATE $T$ := Tour()
    \FORALL {nodes $t_i \in N$}
        \STATE best := $\infty$
        \FORALL {nodes $t_j \in N$}
            \IF {$t_j$ not used \textbf{and} $d_{ij} <$ best}
                \STATE best := $t_j$
                \STATE add(T, best)
            \ENDIF
        \ENDFOR
    \ENDFOR
\end{algorithmic}

\subsection{Slumpmässig}

Ett ytterst snabbt sätt att få fram en fungerande tur är att välja noderna i 
samma ordning som de ligger i indata. I och med att vi inte har några krav på
indata kan ordningen på noderna vara helt slumpmässiga, vi får alltså en 
slumpmässig tur.

Då vi måste gå igenom alla noder kommer tidskomplexiteten bli $O(n)$.


\section{Lokal sökning (optimering)} 

En lokal optimering tar, i fallet med TSP, en redan existerande tur och gör
ändringar i begränsade delar av den för att på så sätt förbättra turens
kvalitet, det vill säga minska dess totala längd.

\subsection{2-opt} 

Om vi tar bort två kanter från en redan existerad tur finns det endast två sätt
att sätta tillbaka dem på. Det ena är samma som tidigare, medan det andra är en
eventuell optimering. Genom att titta på turens längd både innan och efter att
vi flyttade de två kanterna kan vi få reda på om det faktiskt är en optimering
eller ej.
 
\subsubsection{Programbeskrivning}

\begin{algorithmic}
    \STATE $N$ := Nodes()
    \WHILE {moreImprovements}
        \STATE moreImprovements := false
        \FORALL {nodes $t_i \in N$}
            \STATE $t_j$ := next($t_1$)
            \FORALL {nodes $t_k \in N$}
                \STATE $t_l$ := next($t_k$)
                \IF {$d_{ij} + d_{jk} > d_{ij} + d_{jk}$}
                    \STATE reverseBetween(j, k)
                    \STATE moreImprovements := true
                \ENDIF
            \ENDFOR
        \ENDFOR
    \ENDWHILE   
\end{algorithmic}
           

\subsection{2.5-opt} 

%http://www.research.att.com/~dsj/papers/TSPchapter.pdf sida 32källa nummer 21.
%Kanske kan hitta reda på det faktiska pappret på typ lib.kth.se (jag har inget medlemsskap)  

Tekniken som beskrevs i föregående avsnitt kan generaliseras till $k$-opt för
ett fixt tal $k$. Speciellt kan 2-opt göras om till 3-opt där först (inte helt
oväntat) tre kanter i turen tas bort. Detta resulterar i tre isolerade segment,
av vilka alla tillåtna kombinationer evalueras. Den ihopsättning som ger kortast
total längd väljs, innan proceduren upprepas.

Anledningen till att ovanstående tas upp i detta avsnitt är att i 2.5-opt
betraktas de fall av splittringar i 3-opt där ett av de tre segmenten endast
består av en enda nod. Då finns nämligen endast en tillåten kombination utöver
den ursprungliga, vilket erbjuder en billig kontroll av möjligheten till
optimering. Kontrollen går ut på att jämföra längderna av de kanter som tas bort
kontra de kanter som läggs till. Hur detta går till visas utförligare i
programbeskrivningen nedan.

2.5-opt kan kombineras med 2-opt på några olika sätt. Ett sätt är att kombinera
ihop sökningarna så att man för varje kandidat till stad $t_j$ i pseudo-koden
för 2-opt evaluerar motsvarande 2-opt samt även 2.5-opt:en som placerar den
ensamma noden i turen mellan de två varandra intilliggande noderna.

Alternativt kan man låta 2-opt köra tills en lokalt optimal tur har skapats och
först därefter applicera 2.5-opt. Detta visade sig i vårt fall gå snabbare, men
innebar också en smärre försämring av turernas kvalitéer. Eftersom tiden hos
KATTIS är begränsad, valde vi att gå på denna snabbare linje.


\subsubsection{Programbeskrivning}

\begin{algorithmic}
    \STATE $N$ := Nodes()
    \WHILE {improvement}
        \STATE improvment := false
        \FORALL {nodes $t_i \in N$}
            \STATE $t_j$ = next($t_1$)
            \FORALL {nodes $t_k \in N$}
                \STATE $t_l$ = next($t_k$)
                \STATE $t_m$ = next($t_l$)
                \IF {$d(t_i, t_j) + d(t_k, t_l) + d(t_2, t_m) > d(t_4, t_2) + d(t_3, t_2) + d(t_1, t_5)$}
                    \STATE place $t_j$ after $t_k$
                    \STATE improvement := true
                    \STATE restart $t_i$ loop
                \ENDIF
            \ENDFOR
        \ENDFOR
    \ENDWHILE
\end{algorithmic}

\subsection{Optimeringar}

För att snabba upp våra lokalsökningar implementerade vi grannlistor som höll
koll på de närmaste grannarna till varje nod. Det är större sannolikhet att en
granne till en nod kan ge en förbättring i turen än en nod som ligger långt
borta. Vi låter alltså bli att titta på de noder som ligger en bit ifrån den
kant vi håller på att optimera med vår lokala sökning, vilket snabbar upp
körtiden men samtidigt minskar kvaliteten på approximationen något.

När vi gör våra lokala optimeringar beräknar vi inte hela turens längd för att
se om vi har en förbättring eller ej, vi beräknar endast vikterna på de kanter
som flyttas.

Det visade sig vara ganska vanligt att vi vill få reda på vart i turen en viss
nod är, därför implementerade vi en lista som innehåller nodernas position i
listan med noder. Därmed slipper vi linjärsöka igenom hela varje gång vi vill ha
reda på var en nod befinner sig. Kostnaden för att hålla denna lista uppdaterad
betraktade vi som liten i sammanhanget.

Vi valde även att förberäkna alla avstånd och lagra dem i en matris. Vi beräknar
avståndet mellan noder såpass ofta att det visade sig löna sig att strunta i
symetrin hos avstånden, då kollen som behövde ske i distance metoden (en
if-sats) visade sig vara betydligt mycket långsammare än att sätta ytterligare
$n^2/2$ värden i matrisen.



\section{Christofides algoritm} 
%behöver kanske källor till detta? visserligen fick jag
%allt utom tidskomplexiteten från föreläsningen -Linus Wallgren 2009-10-30 00.18

Christofides algoritm har en körtid på $\O(n^3)$ och fungerar på följande vis.

Vi börjar med att skapa ett minimalt spännande träd\footnote{Ett spännande träd
är ett träd som spänner upp alla noder i grafen, medan ett minimalt spännande
träd är det spännande träd som har den minsta totala kantsumman.} i grafen.
Därefter finnes den matchning mellan alla noder av udda grad som har minimal 
total vikt.  

Sedan kombineras den minimala matchningen med det minimala trädet, vars resultat
gås igenom ända tills en Euler-cykel funnits. För att få en Euler-cykel måste vi
hoppa över alla noder som vi redan varit i. När vi sedan har kommit tillbaka
till noden vi startade i har vi fått en hamiltonkrets, d.v.s. en uppskattning på
TSP.  När vi går igenom strukturen kan vi till exempel alltid gå till den
``högraste'' noden, och ifall vi redan har besökt den så lägger vi inte in den i
vår tur igen.

Det är det andra steget i algoritmen, att skapa den minimala matchningen, som
kräver längst tid -- nämligen $\O(n^3)$, vilket dominerar algoritmen i övrigt och
gör den ``långsam''.



\section{Diskussion}

\subsection{Startheuristik}

Det första vi insåg var att startalgoritmen gav stor skillnad på vårt resultat,
trots att vi hade en fungerande 2-opt, som ska ge relativt bra resultat, fick vi
inte alls bra approximationer. Det visade sig vara startheuristiken som var
suboptimal. Istället för att använda sig av Nearest neighbor -- som senare
visade sig vara mycket bättre -- använde vi den av indata givna turen. Efter att
ha justerat detta såg vi drastiska förbättringar i resultatet.

\subsection{Analys}

Vi lyckades implementera alla algoritmer som vi hade tid att försöka oss på.
Våra ursprungliga planer var att hinna implementera 3-opt men dessvärre stötte
vi på tekniska problem vid implementeringen av 2-opt som fördröjde arbetet mer
än vad vi hade räknat med.

För tillfället skalar vårt program väldigt dåligt mot ökad storlek på
grannlistorna, vilket också är ett resultat av förseningen vi upplevde.
Exempelvis kördes vår kod på kattis på $2,5$ sekunder med  en granlista av
storlek 11, ökade vi storleken 12 fick vi istället Time Limit Exceeded, vilket
innebär att det skulle ta mer än 4 sekunder för koden att köra klart, vilket
nästan är en fördubbling av tiden.

Om vi gör samma analys på en lokal dator tar det $0,4$ sekunder att köra klart
ett testfall vi konstruerat med 1000 noder när vi har grannlistor av storlek 15.
Ökar vi istället storleken på grannlistorna till 20 tar det $0,5$ sekunder att
köra klart samma test, vilket ger en helt annan bild än den vi får från Kattis.
Sannolikt är att Kattis har hittat ett testfall som är otroligt elakt mot vår
kod, vilket gör att tiden inte skalar linjärt mot antalet grannar.

Vårt program är uppdelat i ett antal delar och i ett visst testfall med 1000
noder som indata får vi följande resultat, där $n$ är antalet noder och $k$ är
antalet kanter i vår grannmatris.

\begin{tabular}{| l | l | r |}
    \hline
    Operation                               & Tidskomplexitet & Körningstid \\
    \hline \hline
    Läsa in noderna                          & $\O(n)$      & 16,8 \%  \\
    Beräkna alla kantvikter	                 & $\O(n^2)$    & 11,3 \%  \\    
    Hitta närmaste grannarna till var nod    & $\O(n*k)$    & 7.6 \%   \\    
    Nearest Neighbour                        & $\O(n^2)$    & 0.6 \%   \\    
    2-opt	                                 & $\O(n^2*k)$  & 40.9 \%  \\        
    2.5-opt	                                 & $\O(n^2*k)$  & 5,6 \%   \\        
    Skriva ut turen	                         & $\O(n)$      & 0.6 \%   \\    
    Övrigt	                                 & -            & 16.6 \%  \\
    \hline
\end{tabular}


Vi ser att den absolut största delen av tiden går åt till 2-opt. Både den höga
tidskomplexiteten samt det faktum att vi behöver göra väldigt många operationer
(ändra ordning på en hel del element i vår tur) bidrar till att den tar upp så
stor del av tiden.

Att 2.5-opt går såpass fort trots tidskomplexiteten har antagligen att göra med
att de flesta optimeringar redan är gjorda av 2-opten, som körs innan.

Att beräkna alla kantvikter är relativt dyrt, men det lönar sig ganska snart då
kantvikterna är otroligt eftertraktade. Vi ser till exempel att Nearest
neighbour tar väldigt kort tid trots att den har kvadratisk tidskomplexitet.
Detta är i stort tack vare att vi förberäknat resultaten samt även de närmaste
grannarna. 

Framförallt tror vi att vi kan vinna ganska mycket tid genom att optimera koden rent programtekniskt, dvs utan att förändra kvaliteten på resultatet. Vi hade förhoppningen att detta skulle leda till att vi hinner med fler grannar och således får ett bättre resultat alternativt möjligheten att implementera flera optimeringsalgoritmer.

\subsection{Framtida optimeringar} 

Vi skulle kunna implementera en så kallad don't look bit. Om vi inte hittar
några förbättringar för en nod och ingen av deras grannar förändrats, är det
osannolikt att det finns fler förbättringar att göra för den noden. Listan av
don't look-bitar innehåller just den informationen -- huruvida det hänt någon
förändring med vår nod eller dess grannar eller ej. Men även om det minskar
körtiden minskar det också kvaliteten på resultatet, då vi riskerar gå miste om
optimeringar.

Likaså kan man snabba upp koden med en annan datastruktur för själva turen. För
tillfället lagrar vi turen i en vanlig vektor. Om man istället använder sig av
en trädliknande struktur som håller koll på ordningen hos dess barn kommer
operationen \texttt{reverse} att gå betydligt mycket fortare, vilket för
tillfället tar upp det mesta av körtiden när vi kör 2-opt.



\end{document}
